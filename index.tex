% Options for packages loaded elsewhere
\PassOptionsToPackage{unicode}{hyperref}
\PassOptionsToPackage{hyphens}{url}
\PassOptionsToPackage{dvipsnames,svgnames,x11names}{xcolor}
%
\documentclass[
  letterpaper,
  DIV=11,
  numbers=noendperiod]{scrreprt}

\usepackage{amsmath,amssymb}
\usepackage{lmodern}
\usepackage{iftex}
\ifPDFTeX
  \usepackage[T1]{fontenc}
  \usepackage[utf8]{inputenc}
  \usepackage{textcomp} % provide euro and other symbols
\else % if luatex or xetex
  \usepackage{unicode-math}
  \defaultfontfeatures{Scale=MatchLowercase}
  \defaultfontfeatures[\rmfamily]{Ligatures=TeX,Scale=1}
\fi
% Use upquote if available, for straight quotes in verbatim environments
\IfFileExists{upquote.sty}{\usepackage{upquote}}{}
\IfFileExists{microtype.sty}{% use microtype if available
  \usepackage[]{microtype}
  \UseMicrotypeSet[protrusion]{basicmath} % disable protrusion for tt fonts
}{}
\makeatletter
\@ifundefined{KOMAClassName}{% if non-KOMA class
  \IfFileExists{parskip.sty}{%
    \usepackage{parskip}
  }{% else
    \setlength{\parindent}{0pt}
    \setlength{\parskip}{6pt plus 2pt minus 1pt}}
}{% if KOMA class
  \KOMAoptions{parskip=half}}
\makeatother
\usepackage{xcolor}
\setlength{\emergencystretch}{3em} % prevent overfull lines
\setcounter{secnumdepth}{5}
% Make \paragraph and \subparagraph free-standing
\ifx\paragraph\undefined\else
  \let\oldparagraph\paragraph
  \renewcommand{\paragraph}[1]{\oldparagraph{#1}\mbox{}}
\fi
\ifx\subparagraph\undefined\else
  \let\oldsubparagraph\subparagraph
  \renewcommand{\subparagraph}[1]{\oldsubparagraph{#1}\mbox{}}
\fi

\usepackage{color}
\usepackage{fancyvrb}
\newcommand{\VerbBar}{|}
\newcommand{\VERB}{\Verb[commandchars=\\\{\}]}
\DefineVerbatimEnvironment{Highlighting}{Verbatim}{commandchars=\\\{\}}
% Add ',fontsize=\small' for more characters per line
\usepackage{framed}
\definecolor{shadecolor}{RGB}{241,243,245}
\newenvironment{Shaded}{\begin{snugshade}}{\end{snugshade}}
\newcommand{\AlertTok}[1]{\textcolor[rgb]{0.68,0.00,0.00}{#1}}
\newcommand{\AnnotationTok}[1]{\textcolor[rgb]{0.37,0.37,0.37}{#1}}
\newcommand{\AttributeTok}[1]{\textcolor[rgb]{0.40,0.45,0.13}{#1}}
\newcommand{\BaseNTok}[1]{\textcolor[rgb]{0.68,0.00,0.00}{#1}}
\newcommand{\BuiltInTok}[1]{\textcolor[rgb]{0.00,0.23,0.31}{#1}}
\newcommand{\CharTok}[1]{\textcolor[rgb]{0.13,0.47,0.30}{#1}}
\newcommand{\CommentTok}[1]{\textcolor[rgb]{0.37,0.37,0.37}{#1}}
\newcommand{\CommentVarTok}[1]{\textcolor[rgb]{0.37,0.37,0.37}{\textit{#1}}}
\newcommand{\ConstantTok}[1]{\textcolor[rgb]{0.56,0.35,0.01}{#1}}
\newcommand{\ControlFlowTok}[1]{\textcolor[rgb]{0.00,0.23,0.31}{#1}}
\newcommand{\DataTypeTok}[1]{\textcolor[rgb]{0.68,0.00,0.00}{#1}}
\newcommand{\DecValTok}[1]{\textcolor[rgb]{0.68,0.00,0.00}{#1}}
\newcommand{\DocumentationTok}[1]{\textcolor[rgb]{0.37,0.37,0.37}{\textit{#1}}}
\newcommand{\ErrorTok}[1]{\textcolor[rgb]{0.68,0.00,0.00}{#1}}
\newcommand{\ExtensionTok}[1]{\textcolor[rgb]{0.00,0.23,0.31}{#1}}
\newcommand{\FloatTok}[1]{\textcolor[rgb]{0.68,0.00,0.00}{#1}}
\newcommand{\FunctionTok}[1]{\textcolor[rgb]{0.28,0.35,0.67}{#1}}
\newcommand{\ImportTok}[1]{\textcolor[rgb]{0.00,0.46,0.62}{#1}}
\newcommand{\InformationTok}[1]{\textcolor[rgb]{0.37,0.37,0.37}{#1}}
\newcommand{\KeywordTok}[1]{\textcolor[rgb]{0.00,0.23,0.31}{#1}}
\newcommand{\NormalTok}[1]{\textcolor[rgb]{0.00,0.23,0.31}{#1}}
\newcommand{\OperatorTok}[1]{\textcolor[rgb]{0.37,0.37,0.37}{#1}}
\newcommand{\OtherTok}[1]{\textcolor[rgb]{0.00,0.23,0.31}{#1}}
\newcommand{\PreprocessorTok}[1]{\textcolor[rgb]{0.68,0.00,0.00}{#1}}
\newcommand{\RegionMarkerTok}[1]{\textcolor[rgb]{0.00,0.23,0.31}{#1}}
\newcommand{\SpecialCharTok}[1]{\textcolor[rgb]{0.37,0.37,0.37}{#1}}
\newcommand{\SpecialStringTok}[1]{\textcolor[rgb]{0.13,0.47,0.30}{#1}}
\newcommand{\StringTok}[1]{\textcolor[rgb]{0.13,0.47,0.30}{#1}}
\newcommand{\VariableTok}[1]{\textcolor[rgb]{0.07,0.07,0.07}{#1}}
\newcommand{\VerbatimStringTok}[1]{\textcolor[rgb]{0.13,0.47,0.30}{#1}}
\newcommand{\WarningTok}[1]{\textcolor[rgb]{0.37,0.37,0.37}{\textit{#1}}}

\providecommand{\tightlist}{%
  \setlength{\itemsep}{0pt}\setlength{\parskip}{0pt}}\usepackage{longtable,booktabs,array}
\usepackage{calc} % for calculating minipage widths
% Correct order of tables after \paragraph or \subparagraph
\usepackage{etoolbox}
\makeatletter
\patchcmd\longtable{\par}{\if@noskipsec\mbox{}\fi\par}{}{}
\makeatother
% Allow footnotes in longtable head/foot
\IfFileExists{footnotehyper.sty}{\usepackage{footnotehyper}}{\usepackage{footnote}}
\makesavenoteenv{longtable}
\usepackage{graphicx}
\makeatletter
\def\maxwidth{\ifdim\Gin@nat@width>\linewidth\linewidth\else\Gin@nat@width\fi}
\def\maxheight{\ifdim\Gin@nat@height>\textheight\textheight\else\Gin@nat@height\fi}
\makeatother
% Scale images if necessary, so that they will not overflow the page
% margins by default, and it is still possible to overwrite the defaults
% using explicit options in \includegraphics[width, height, ...]{}
\setkeys{Gin}{width=\maxwidth,height=\maxheight,keepaspectratio}
% Set default figure placement to htbp
\makeatletter
\def\fps@figure{htbp}
\makeatother

\KOMAoption{captions}{tableheading}
\makeatletter
\makeatother
\makeatletter
\@ifpackageloaded{bookmark}{}{\usepackage{bookmark}}
\makeatother
\makeatletter
\@ifpackageloaded{caption}{}{\usepackage{caption}}
\AtBeginDocument{%
\ifdefined\contentsname
  \renewcommand*\contentsname{Table of contents}
\else
  \newcommand\contentsname{Table of contents}
\fi
\ifdefined\listfigurename
  \renewcommand*\listfigurename{List of Figures}
\else
  \newcommand\listfigurename{List of Figures}
\fi
\ifdefined\listtablename
  \renewcommand*\listtablename{List of Tables}
\else
  \newcommand\listtablename{List of Tables}
\fi
\ifdefined\figurename
  \renewcommand*\figurename{Figure}
\else
  \newcommand\figurename{Figure}
\fi
\ifdefined\tablename
  \renewcommand*\tablename{Table}
\else
  \newcommand\tablename{Table}
\fi
}
\@ifpackageloaded{float}{}{\usepackage{float}}
\floatstyle{ruled}
\@ifundefined{c@chapter}{\newfloat{codelisting}{h}{lop}}{\newfloat{codelisting}{h}{lop}[chapter]}
\floatname{codelisting}{Listing}
\newcommand*\listoflistings{\listof{codelisting}{List of Listings}}
\makeatother
\makeatletter
\@ifpackageloaded{caption}{}{\usepackage{caption}}
\@ifpackageloaded{subcaption}{}{\usepackage{subcaption}}
\makeatother
\makeatletter
\@ifpackageloaded{tcolorbox}{}{\usepackage[many]{tcolorbox}}
\makeatother
\makeatletter
\@ifundefined{shadecolor}{\definecolor{shadecolor}{rgb}{.97, .97, .97}}
\makeatother
\makeatletter
\makeatother
\ifLuaTeX
  \usepackage{selnolig}  % disable illegal ligatures
\fi
\IfFileExists{bookmark.sty}{\usepackage{bookmark}}{\usepackage{hyperref}}
\IfFileExists{xurl.sty}{\usepackage{xurl}}{} % add URL line breaks if available
\urlstyle{same} % disable monospaced font for URLs
\hypersetup{
  pdftitle={{[}STUDENT/GROUP NAME{]} Progress Journal},
  colorlinks=true,
  linkcolor={blue},
  filecolor={Maroon},
  citecolor={Blue},
  urlcolor={Blue},
  pdfcreator={LaTeX via pandoc}}

\title{{[}STUDENT/GROUP NAME{]} Progress Journal}
\author{}
\date{}

\begin{document}
\maketitle
\ifdefined\Shaded\renewenvironment{Shaded}{\begin{tcolorbox}[frame hidden, interior hidden, borderline west={3pt}{0pt}{shadecolor}, enhanced, breakable, boxrule=0pt, sharp corners]}{\end{tcolorbox}}\fi

\renewcommand*\contentsname{Table of contents}
{
\hypersetup{linkcolor=}
\setcounter{tocdepth}{2}
\tableofcontents
}
\bookmarksetup{startatroot}

\hypertarget{introduction}{%
\chapter*{Introduction}\label{introduction}}
\addcontentsline{toc}{chapter}{Introduction}

This progress journal covers {[}STUDENT NAME SURNAME / PROJECT GROUP
NAME{]}'s work during their term at
\href{https://mef-bda503.github.io/fall22/}{BDA 503 Fall 2022}.

Each section is an assignment or an individual work.

\bookmarksetup{startatroot}

\hypertarget{bda-503-assigment-1}{%
\chapter{BDA-503 Assigment 1}\label{bda-503-assigment-1}}

A short brief of author and R use cases

Emre Çakmak\\
2022-10-09

\hfill\break

Hi dear reader,

I'm Emre Çakmak from Istanbul/Turkey. I graduated from my bachelor at
Istanbul Technical University, Industrial Engineering Department in
2018.

My current role is Data Scientist at E-commerce Department in LC Waikiki
which is a Istanbul based global fashion retailer driving operations on
more than 50 countries. I had different positions like Data Analyst,
Business Intelligence Specialist in different companies during past 4
years. Especially in last 1 year, I dedicated to improve myself for
application of ML Technics due to enrich customer\&item based data. So,
I'm a part of BDA Graduate Program in MEF University to wide my
knowledge in audience management and marketing applications by the help
of real-life use cases.

\href{https://www.linkedin.com/in/emre-\%C3\%A7akmak-7778b160/}{\textbf{Here
is my LinkedIn Profile}}

\begin{figure}

\href{https://www.linkedin.com/in/emre-\%C3\%A7akmak-7778b160/}{\includegraphics[width=1.04167in,height=\textheight]{./images/emrecakmak.png}}

\end{figure}

\hypertarget{rstudio-global-2022-conference---quarto-for-the-curious}{%
\section{RStudio Global 2022 Conference - Quarto for the
Curious}\label{rstudio-global-2022-conference---quarto-for-the-curious}}

\href{https://www.rstudio.com/conference/2022/talks/quarto-for-rmarkdown-users/}{What's
\emph{Quarto} according to Tom Mock}

In this paragraph, I aim to give you some main differences between
\emph{Quarto,} the brand new documentation system which has been
released April 2022, and \emph{RMarkdown} being used for almost a
decade.

\begin{itemize}
\tightlist
\item
  Tom Mock says \emph{Quarto} is Open source scientific and technical
  publishing system. Also he added that \emph{Quarto} is the next
  generation of \emph{RMarkdown}.
\end{itemize}

Here is some differences between them:

\hypertarget{preprocessing}{%
\subsection{Preprocessing}\label{preprocessing}}

\begin{figure}

\begin{minipage}[t]{0.50\linewidth}

{\centering 

\raisebox{-\height}{

\includegraphics[width=3.64583in,height=\textheight]{./images/rmarkdown.PNG}

}

}

\subcaption{\label{fig-rmarkdown}RMarkdown}
\end{minipage}%
%
\begin{minipage}[t]{0.50\linewidth}

{\centering 

\raisebox{-\height}{

\includegraphics[width=3.64583in,height=\textheight]{./images/quatro.PNG}

}

}

\subcaption{\label{fig-quatro}Quarto}
\end{minipage}%

\caption{\label{fig-quarto-vs-rmarkdown}RMarkdown vs Quarto
Preprocessing Diagram}

\end{figure}

Altough it seems like they have almost same workflow behind the scenes;
\emph{Quarto} doesn't need to have R in the system to use it. It means
that you can use \emph{Quarto} in a fresh computer but \emph{Rmarkdown}
needs to have R in the system.

\hypertarget{language-support}{%
\subsection{Language Support}\label{language-support}}

The main purpose of releasing \emph{Quarto} is improving the
communication between data science communities whatever their language
is. Because of this \emph{Quarto} supports other languages as engine.

\begin{figure}

{\centering 

\includegraphics[width=3.64583in,height=\textheight]{./images/quartojupytersupport.PNG}

}

\caption{\label{fig-quarto-language-support}Jupyter as Quarto Engine}

\end{figure}

This availability in \emph{Quarto} and not limiting with R allows people
to collaborate as Python developer with others. Tom Mock figured this
situation out like

\begin{itemize}
\tightlist
\item
  \textbf{Quarto: Comfortable baking in your own kitchen}
\item
  \textbf{RMarkdown: Uncomfortable baking in corporate kitchen}
\end{itemize}

\hypertarget{r-posts}{%
\section{R Posts}\label{r-posts}}

This section includes 3 different R Programming use case

\hypertarget{web-scraping-with-r}{%
\subsection{Web Scraping with R}\label{web-scraping-with-r}}

It's very known fact that people have some struggle to access to a clean
dataset. In these cases, we need to be a little bit creative to create
our own dataset. And one way of the creating a new dataset is web
scraping.

In this paragraph, I want to introduce how to scrape a web page by the
help of R packages. The most common 2 packages are:

\begin{itemize}
\tightlist
\item
  \{rvest\}
\item
  \{RSelenium\}
\end{itemize}

\textbf{Note that: Some websites have strict policies against scraping.
Be careful!}

Step by step scraping of public IMDB Dataset

\textbf{Step 1: Install Package}

\begin{Shaded}
\begin{Highlighting}[]
\DocumentationTok{\#\# Before you start, you need to execute once the code below.}
\DocumentationTok{\#\#install.packages("rvest")}
\end{Highlighting}
\end{Shaded}

\textbf{Step 2: Call the library and use html functions}

\begin{Shaded}
\begin{Highlighting}[]
\DocumentationTok{\#\# call the rvest library for required functions}
\FunctionTok{library}\NormalTok{(rvest)}

\DocumentationTok{\#\# define the website link you want to scrape}
\NormalTok{link }\OtherTok{=} \StringTok{"https://www.imdb.com/search/title/?title\_type=feature\&num\_votes=30000,\&genres=comedy"}

\DocumentationTok{\#\# send a http get request to the link above and store it in a variable}
\NormalTok{page }\OtherTok{=} \FunctionTok{read\_html}\NormalTok{(link)}

\DocumentationTok{\#\# filter and grab all elements in same class}
\NormalTok{titles }\OtherTok{=}\NormalTok{ page }\SpecialCharTok{\%\textgreater{}\%} \FunctionTok{html\_nodes}\NormalTok{(}\StringTok{".lister{-}item{-}header a"}\NormalTok{) }\SpecialCharTok{\%\textgreater{}\%} \FunctionTok{html\_text}\NormalTok{()}

\DocumentationTok{\#\# preview the titles}
\NormalTok{titles[}\DecValTok{1}\SpecialCharTok{:}\DecValTok{10}\NormalTok{]}
\end{Highlighting}
\end{Shaded}

\begin{verbatim}
 [1] "Hocus Pocus"                       "Bullet Train"                     
 [3] "Thor: Love and Thunder"            "DC League of Super-Pets"          
 [5] "Everything Everywhere All at Once" "Beetle Juice"                     
 [7] "Super Mario Bros."                 "Trick 'r Treat"                   
 [9] "Once Upon a Time... in Hollywood"  "Knives Out"                       
\end{verbatim}

\textbf{Step 3: Create other variables}

\begin{Shaded}
\begin{Highlighting}[]
\DocumentationTok{\#\# apply same procedure to other variables}
\NormalTok{year}\OtherTok{=}\NormalTok{ page }\SpecialCharTok{\%\textgreater{}\%} \FunctionTok{html\_nodes}\NormalTok{(}\StringTok{".text{-}muted.unbold"}\NormalTok{) }\SpecialCharTok{\%\textgreater{}\%} \FunctionTok{html\_text}\NormalTok{()}
\NormalTok{rating }\OtherTok{=}\NormalTok{ page }\SpecialCharTok{\%\textgreater{}\%} \FunctionTok{html\_nodes}\NormalTok{(}\StringTok{".ratings{-}imdb{-}rating strong"}\NormalTok{) }\SpecialCharTok{\%\textgreater{}\%} \FunctionTok{html\_text}\NormalTok{()}


\DocumentationTok{\#\# preview variables}
\NormalTok{year[}\DecValTok{1}\SpecialCharTok{:}\DecValTok{10}\NormalTok{]}
\end{Highlighting}
\end{Shaded}

\begin{verbatim}
 [1] "(1993)" "(2022)" "(2022)" "(2022)" "(2022)" "(1988)" "(1993)" "(2007)"
 [9] "(2019)" "(2019)"
\end{verbatim}

\begin{Shaded}
\begin{Highlighting}[]
\NormalTok{rating[}\DecValTok{1}\SpecialCharTok{:}\DecValTok{10}\NormalTok{]}
\end{Highlighting}
\end{Shaded}

\begin{verbatim}
 [1] "6.9" "7.4" "6.4" "7.4" "8.1" "7.5" "4.1" "6.7" "7.6" "7.9"
\end{verbatim}

\textbf{Step 4: Create data frame}

\begin{Shaded}
\begin{Highlighting}[]
\DocumentationTok{\#\# create a dataset}
\NormalTok{movies }\OtherTok{=} \FunctionTok{data.frame}\NormalTok{(titles, year, rating, }\AttributeTok{stringsAsFactors =} \ConstantTok{FALSE}\NormalTok{)}
\NormalTok{movies[}\DecValTok{1}\SpecialCharTok{:}\DecValTok{10}\NormalTok{,]}
\end{Highlighting}
\end{Shaded}

\begin{verbatim}
                              titles   year rating
1                        Hocus Pocus (1993)    6.9
2                       Bullet Train (2022)    7.4
3             Thor: Love and Thunder (2022)    6.4
4            DC League of Super-Pets (2022)    7.4
5  Everything Everywhere All at Once (2022)    8.1
6                       Beetle Juice (1988)    7.5
7                  Super Mario Bros. (1993)    4.1
8                     Trick 'r Treat (2007)    6.7
9   Once Upon a Time... in Hollywood (2019)    7.6
10                        Knives Out (2019)    7.9
\end{verbatim}

References of web scraping with R:

\begin{itemize}
\tightlist
\item
  \href{https://www.scraperapi.com/blog/web-scraping-with-r/}{Scraperapi}
\item
  \href{https://www.scrapingbee.com/blog/web-scraping-r/}{Scrapingbee}
\item
  \href{https://appsilon.com/webscraping-dynamic-websites-with-r/}{Appsilon}
\end{itemize}

\hypertarget{simple-aggregations-on-dataset}{%
\subsection{Simple Aggregations on
Dataset}\label{simple-aggregations-on-dataset}}

This part provides some basic aggregations and data manipulation methods
in R via \{dplyr\} package.

Without leaving the concept in previous part, we can assume that we
created our own dataset. So, what's next?

The process of extracting insightful information from datasets starts
from understanding the data structure and manipulating them. R provides
a package just for this: \textbf{\{dplyr\}}

Step by step aggregation \& filtering \& summarizing dataset

\textbf{Step 1: Install Package}

\begin{Shaded}
\begin{Highlighting}[]
\DocumentationTok{\#\# Before you start, you need to execute once the code below.}
\DocumentationTok{\#\#install.packages("dplyr")}
\end{Highlighting}
\end{Shaded}

\textbf{Step 2: Call the library}

\begin{Shaded}
\begin{Highlighting}[]
\FunctionTok{library}\NormalTok{(dplyr)}
\end{Highlighting}
\end{Shaded}

\begin{verbatim}

Attaching package: 'dplyr'
\end{verbatim}

\begin{verbatim}
The following objects are masked from 'package:stats':

    filter, lag
\end{verbatim}

\begin{verbatim}
The following objects are masked from 'package:base':

    intersect, setdiff, setequal, union
\end{verbatim}

\textbf{Step 3: Select subset of data in different aspects}

\begin{Shaded}
\begin{Highlighting}[]
\DocumentationTok{\#\# selecting specific columns}
\FunctionTok{select}\NormalTok{(movies, titles, year)[}\DecValTok{1}\SpecialCharTok{:}\DecValTok{10}\NormalTok{,]}
\end{Highlighting}
\end{Shaded}

\begin{verbatim}
                              titles   year
1                        Hocus Pocus (1993)
2                       Bullet Train (2022)
3             Thor: Love and Thunder (2022)
4            DC League of Super-Pets (2022)
5  Everything Everywhere All at Once (2022)
6                       Beetle Juice (1988)
7                  Super Mario Bros. (1993)
8                     Trick 'r Treat (2007)
9   Once Upon a Time... in Hollywood (2019)
10                        Knives Out (2019)
\end{verbatim}

\begin{Shaded}
\begin{Highlighting}[]
\DocumentationTok{\#\# filter data according to specific condition}
\FunctionTok{filter}\NormalTok{(movies, rating }\SpecialCharTok{\textgreater{}} \DecValTok{8}\NormalTok{)}
\end{Highlighting}
\end{Shaded}

\begin{verbatim}
                             titles   year rating
1 Everything Everywhere All at Once (2022)    8.1
2           The Wolf of Wall Street (2013)    8.2
3                          Deadpool (2016)    8.0
4                Back to the Future (1985)    8.5
\end{verbatim}

\begin{Shaded}
\begin{Highlighting}[]
\DocumentationTok{\#\# sort rows}
\FunctionTok{arrange}\NormalTok{(movies, }\FunctionTok{desc}\NormalTok{(titles))[}\DecValTok{1}\SpecialCharTok{:}\DecValTok{10}\NormalTok{,]}
\end{Highlighting}
\end{Shaded}

\begin{verbatim}
                                    titles   year rating
1                           Trick 'r Treat (2007)    6.7
2                   Thor: Love and Thunder (2022)    6.4
3                  The Wolf of Wall Street (2013)    8.2
4                              The Witches (1990)    6.8
5  The Unbearable Weight of Massive Talent (2022)    7.0
6                        The Suicide Squad (2021)    7.2
7            The Rocky Horror Picture Show (1975)    7.4
8                            The Lost City (2022)    6.1
9                            The Lost Boys (1987)    7.2
10                             The Goonies (1985)    7.7
\end{verbatim}

\begin{Shaded}
\begin{Highlighting}[]
\DocumentationTok{\#\# select top n rows}
\FunctionTok{top\_n}\NormalTok{(movies, }\DecValTok{3}\NormalTok{, titles)}
\end{Highlighting}
\end{Shaded}

\begin{verbatim}
                   titles   year rating
1  Thor: Love and Thunder (2022)    6.4
2          Trick 'r Treat (2007)    6.7
3 The Wolf of Wall Street (2013)    8.2
\end{verbatim}

\textbf{Step 4: Summarize Dataset}

\begin{Shaded}
\begin{Highlighting}[]
\DocumentationTok{\#\# convert rating columns as numeric and calculate the average}
\FunctionTok{summarise}\NormalTok{(movies, }\AttributeTok{average\_rating =} \FunctionTok{mean}\NormalTok{(}\FunctionTok{as.numeric}\NormalTok{(rating)))}
\end{Highlighting}
\end{Shaded}

\begin{verbatim}
  average_rating
1          6.882
\end{verbatim}

\begin{Shaded}
\begin{Highlighting}[]
\DocumentationTok{\#\# group by and summarize}
\NormalTok{grouped\_data }\OtherTok{=} \FunctionTok{group\_by}\NormalTok{(movies, year)}
\FunctionTok{summarise}\NormalTok{(grouped\_data, }\AttributeTok{average\_rating =} \FunctionTok{mean}\NormalTok{(}\FunctionTok{as.numeric}\NormalTok{(rating)))[}\DecValTok{1}\SpecialCharTok{:}\DecValTok{5}\NormalTok{,]}
\end{Highlighting}
\end{Shaded}

\begin{verbatim}
# A tibble: 5 x 2
  year   average_rating
  <chr>           <dbl>
1 (1975)            7.4
2 (1984)            7.8
3 (1985)            8.1
4 (1987)            7.2
5 (1988)            7.5
\end{verbatim}

\textbf{Step 5: \%\textgreater\% Operator}

This operator takes the object from the left and gives it as the first
argument to the function on the right. It makes your code more readable.

\begin{Shaded}
\begin{Highlighting}[]
\DocumentationTok{\#\# same grouping and summarizing operation at step4 }

\NormalTok{movies }\SpecialCharTok{\%\textgreater{}\%}
  \FunctionTok{group\_by}\NormalTok{(year) }\SpecialCharTok{\%\textgreater{}\%}
  \FunctionTok{summarise}\NormalTok{(}\AttributeTok{average\_rating =} \FunctionTok{mean}\NormalTok{(}\FunctionTok{as.numeric}\NormalTok{(rating)))}\SpecialCharTok{\%\textgreater{}\%}
  \FunctionTok{top\_n}\NormalTok{(}\DecValTok{5}\NormalTok{, }\FunctionTok{desc}\NormalTok{(average\_rating))}
\end{Highlighting}
\end{Shaded}

\begin{verbatim}
# A tibble: 7 x 2
  year       average_rating
  <chr>               <dbl>
1 (1993)                5.5
2 (1995)                6.5
3 (2002)                5.2
4 (2005)                6.5
5 (2009)                5.3
6 (2020)                5.2
7 (I) (2022)            6.5
\end{verbatim}

Reference of aggregations with R:

\begin{itemize}
\tightlist
\item
  \href{https://courses.cs.ut.ee/MTAT.03.183/2017_fall/uploads/Main/dplyr.html}{courses.cs.ut.ee}
\end{itemize}

\hypertarget{visualization-with-r}{%
\subsection{Visualization with R}\label{visualization-with-r}}

\textbf{Step 1: Install Package}

\begin{Shaded}
\begin{Highlighting}[]
\DocumentationTok{\#\# Before you start, you need to execute once the code below.}
\DocumentationTok{\#\#install.packages("ggplot2")}
\end{Highlighting}
\end{Shaded}

\textbf{Step 2: Call the library}

\begin{Shaded}
\begin{Highlighting}[]
\FunctionTok{library}\NormalTok{(ggplot2)}
\end{Highlighting}
\end{Shaded}

\textbf{Step 3: Histogram with ggplot2}

\begin{Shaded}
\begin{Highlighting}[]
\DocumentationTok{\#\# convert rating field as numeric and keep in original dataset}
\NormalTok{movies}\SpecialCharTok{$}\NormalTok{rating}\OtherTok{=}\FunctionTok{as.numeric}\NormalTok{(rating)}

\DocumentationTok{\#\# histogram for ratings}
\FunctionTok{hist}\NormalTok{(movies}\SpecialCharTok{$}\NormalTok{rating,}\AttributeTok{col=}\StringTok{\textquotesingle{}steelblue\textquotesingle{}}\NormalTok{,}\AttributeTok{main=}\StringTok{\textquotesingle{}Rating Histogram\textquotesingle{}}\NormalTok{,}
     \AttributeTok{xlab=}\StringTok{\textquotesingle{}Ratings\textquotesingle{}}\NormalTok{)}
\end{Highlighting}
\end{Shaded}

\begin{figure}[H]

{\centering \includegraphics{./assignment1_files/figure-pdf/visualization-chunk3-1.pdf}

}

\end{figure}

\textbf{Step 4: Pie Chart with ggplot2}

\begin{Shaded}
\begin{Highlighting}[]
\DocumentationTok{\#\# Set a new flag in dataset}
\NormalTok{movies}\OtherTok{=}\NormalTok{movies }\SpecialCharTok{\%\textgreater{}\%} \FunctionTok{mutate}\NormalTok{(}\AttributeTok{rating\_flag =} \FunctionTok{case\_when}\NormalTok{(rating}\SpecialCharTok{\textgreater{}=}\DecValTok{8}\SpecialCharTok{\textasciitilde{}} \StringTok{"Higher 8"}\NormalTok{, }\ConstantTok{TRUE} \SpecialCharTok{\textasciitilde{}} \StringTok{"Lower 8"}\NormalTok{))}

\DocumentationTok{\#\# creating a new table for better visualization}
\NormalTok{count\_movies}\OtherTok{=}\NormalTok{movies }\SpecialCharTok{\%\textgreater{}\%} \FunctionTok{count}\NormalTok{(rating\_flag)}

\DocumentationTok{\#\# pie chart according to rating of movies}
\FunctionTok{ggplot}\NormalTok{(count\_movies, }\FunctionTok{aes}\NormalTok{(}\AttributeTok{x =} \StringTok{""}\NormalTok{, }\AttributeTok{y =}\NormalTok{ n, }\AttributeTok{fill =}\NormalTok{ rating\_flag)) }\SpecialCharTok{+}
  \FunctionTok{geom\_col}\NormalTok{(}\AttributeTok{color =} \StringTok{"black"}\NormalTok{) }\SpecialCharTok{+}
  \FunctionTok{geom\_label}\NormalTok{(}\FunctionTok{aes}\NormalTok{(}\AttributeTok{label =}\NormalTok{ n),}
             \AttributeTok{position =} \FunctionTok{position\_stack}\NormalTok{(}\AttributeTok{vjust =} \FloatTok{0.5}\NormalTok{),}
             \AttributeTok{show.legend =} \ConstantTok{FALSE}\NormalTok{) }\SpecialCharTok{+}
  \FunctionTok{coord\_polar}\NormalTok{(}\AttributeTok{theta =} \StringTok{"y"}\NormalTok{)}
\end{Highlighting}
\end{Shaded}

\begin{figure}[H]

{\centering \includegraphics{./assignment1_files/figure-pdf/visualization-chunk4-1.pdf}

}

\end{figure}

References of visualization with R:

\begin{itemize}
\tightlist
\item
  \href{https://r-charts.com/part-whole/pie-chart-ggplot2/}{r-charts}
\item
  \href{https://www.kdnuggets.com/2018/06/7-simple-data-visualizations-should-know-r.html}{kdnuggets}
\end{itemize}



\end{document}
